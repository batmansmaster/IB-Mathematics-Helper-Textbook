\chapter{Properties of Logarithms}

\section{Laws of logarithms}
Last unit we've covered exponential and logarithmic functions, which basically govern a lot of our daily phenomena. We've solved simple exponential functions, such as $2^x=32\implies x=5$ because $32=2^5$
But the equation doesn't have to be that nice all the time. What would you do if you are asked to solve $3^x=5$? You can't write 5 as a power of 3. There must be other ways to solve these types of equations, right?

The laws of logarithms are based on the rules of indices. If you're quite shaky on rules of indices, go back to unit 1 and do a handful of exercises to brush up your skills. You are expected to do the problems in that section without the use of a calculator.

Let's briefly recap the definition of logarithm:
\[b=a^x \iff x=\log_ab\]
The function is defined where $b>0$ and $x\geq 0$.

From the definition of logarithm, we know that $\log_b(b^x)=x$ for every real values of $x$.
Think about it for a bit. You're asked to find the power of b in $b^x$. The answer is obviously $x$.

Let's get to the three rules of logarithms.

\vspace{5mm}
\par\textbf{Law 1}
\par Product rule.
\par The logarithm of a \textit{product} is the \textit{sum} of the logarithms.

{\hfill\Large\bfseries NEEDS FIXING\hfill}
\begin{lstlisting}
\begin{formula}
\log_a(xy) = \log_a(x) + \log_a(y)
\end{formula}
 \end{lstlisting}


\vspace{5mm}
\par\textbf{Law 2}
\par Division rule.
\par The logarithm of a \textit{quotient} is the \textit{difference} of the logarithms.


{\hfill\Large\bfseries NEEDS FIXING\hfill}
\begin{lstlisting}
\begin{formula}
\log_a\left(\frac{x}{y}\right)=\log_a(x)-\log_a(y)
\end{formula}
 \end{lstlisting}


\vspace{5mm}
\par \textbf{Law 3}
\par Consequence of law 1
\par The logarithm of a \textit{exponent} is the \textit {multiple} of the logarithm.

{\hfill\Large\bfseries NEEDS FIXING\hfill}
\begin{lstlisting}
\begin{formula}
\log_a(x^r) = r\log_a(x)
\end{formula}
 \end{lstlisting}

\vspace{5mm}
The same thing applies to natural logarithm, $\ln x$.
Recall that $\log_e(x)=\ln x$ and $\ln(e)=1$, consequently, we have

{\hfill\Large\bfseries NEEDS FIXING\hfill}
\begin{lstlisting}
\begin{formula}
\ln(e^x) = x
\end{formula}
 \end{lstlisting}

\begin{example}{Logarithms}
Express \(\log_3125\) in terms of \(a\) and \(b\), where \(\log_25=a\) and \(\log_23=b\).

\sep
\begin{align}
    \log_3125=\log_35^{3}&=3\log_35\\
    \log_35=\frac{\log_25}{\log_23}&=\frac{a}{b}\\
    \implies \log_3125&=\frac{3a}{b}
\end{align}
\end{example}
