\documentclass[../../main.tex]{subfiles}

\begin{document}

\chapter{Graph Theory (HL)}

Graph theory is a topic that's unrelated to everything else in the syllabus, making it a unique topic that requires a different way of thinking and solving problems.
Fortunately, this means that studying graph theory doesn't technically have any prerequisites and can be taken at any time.
What makes it even easier to learn this topic is that graph theory has a lot of applications in real life which are direct and easy to understand, which makes concepts much more intuitive.

\section{Terminologies}
Seeing that the topic is new for most students, we're going to start with some of the most common definitions in graph theory.
An informal definition of a graph is that it's basically a structure that we use to represent items of interests and how those items are connected.

\begin{definition}{Graph}
A graph is a set of vertices and edges.
We denote a graph by a capital letter; most commonly we use G or H (if G's already a graph).
We write that
\begin{equation}
    G = \{V, E\}
\end{equation}
where $V$ is the set of vertices and $E$ is the set of edges.
%Insert an example of a graph here
\end{definition}

\begin{definition}{Vertex}
A vertex (plural vertices), also called node, is a ``point'' on the graph.
The set of vertices in a graph is usually denoted as $V$.
When a vertex $v$ is part of a graph we denote that by $v \in V$.

\textbf{Note:} the term vertex and node mean the same thing and they'll be used interchangeably.
\end{definition}

\begin{definition}{Edge}
An edge is a ``line'' connecting 2 vertices (exception will be shown later).
The set of edges in a graph is usally denoted as $E$.
An edge can be denoted in different ways; it can be denoted as $e_n$ or according to the vertices at both ends (like a segment in a triangle).
When an edge $e_n$ is part of the edge set we denote that by $e_n \in E$.
%Insert pic
\end{definition}

\end{document}